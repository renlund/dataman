%\VignetteEngine{knitr::knitr}
%\VignetteIndexEntry{Example of 'cr_group' and its latex method}
\documentclass{article}\usepackage[]{graphicx}\usepackage[]{color}
%% maxwidth is the original width if it is less than linewidth
%% otherwise use linewidth (to make sure the graphics do not exceed the margin)
\makeatletter
\def\maxwidth{ %
  \ifdim\Gin@nat@width>\linewidth
    \linewidth
  \else
    \Gin@nat@width
  \fi
}
\makeatother

\definecolor{fgcolor}{rgb}{0.345, 0.345, 0.345}
\newcommand{\hlnum}[1]{\textcolor[rgb]{0.686,0.059,0.569}{#1}}%
\newcommand{\hlstr}[1]{\textcolor[rgb]{0.192,0.494,0.8}{#1}}%
\newcommand{\hlcom}[1]{\textcolor[rgb]{0.678,0.584,0.686}{\textit{#1}}}%
\newcommand{\hlopt}[1]{\textcolor[rgb]{0,0,0}{#1}}%
\newcommand{\hlstd}[1]{\textcolor[rgb]{0.345,0.345,0.345}{#1}}%
\newcommand{\hlkwa}[1]{\textcolor[rgb]{0.161,0.373,0.58}{\textbf{#1}}}%
\newcommand{\hlkwb}[1]{\textcolor[rgb]{0.69,0.353,0.396}{#1}}%
\newcommand{\hlkwc}[1]{\textcolor[rgb]{0.333,0.667,0.333}{#1}}%
\newcommand{\hlkwd}[1]{\textcolor[rgb]{0.737,0.353,0.396}{\textbf{#1}}}%

\usepackage{framed}
\makeatletter
\newenvironment{kframe}{%
 \def\at@end@of@kframe{}%
 \ifinner\ifhmode%
  \def\at@end@of@kframe{\end{minipage}}%
  \begin{minipage}{\columnwidth}%
 \fi\fi%
 \def\FrameCommand##1{\hskip\@totalleftmargin \hskip-\fboxsep
 \colorbox{shadecolor}{##1}\hskip-\fboxsep
     % There is no \\@totalrightmargin, so:
     \hskip-\linewidth \hskip-\@totalleftmargin \hskip\columnwidth}%
 \MakeFramed {\advance\hsize-\width
   \@totalleftmargin\z@ \linewidth\hsize
   \@setminipage}}%
 {\par\unskip\endMakeFramed%
 \at@end@of@kframe}
\makeatother

\definecolor{shadecolor}{rgb}{.97, .97, .97}
\definecolor{messagecolor}{rgb}{0, 0, 0}
\definecolor{warningcolor}{rgb}{1, 0, 1}
\definecolor{errorcolor}{rgb}{1, 0, 0}
\newenvironment{knitrout}{}{} % an empty environment to be redefined in TeX

\usepackage{alltt}

\title{S3 class \texttt{cr\_group} and its \texttt{latex} method}
\author{Henrik Renlund}
\IfFileExists{upquote.sty}{\usepackage{upquote}}{}
\begin{document}
\maketitle
\tableofcontents



\setcounter{section}{-1}
\section{Excuses, excuses}
This vignette is a mess\ldots

\section{Generation of data}
\begin{knitrout}
\definecolor{shadecolor}{rgb}{0.969, 0.969, 0.969}\color{fgcolor}\begin{kframe}
\begin{alltt}
\hlstd{nr} \hlkwb{<-} \hlnum{7}\hlstd{; nc} \hlkwb{<-} \hlnum{5}\hlstd{; m} \hlkwb{<-} \hlkwd{matrix}\hlstd{(}\hlnum{1}\hlopt{:}\hlstd{(nr}\hlopt{*}\hlstd{nc),} \hlkwc{nrow}\hlstd{=nr,} \hlkwc{byrow}\hlstd{=}\hlnum{TRUE}\hlstd{)}
\hlkwd{rownames}\hlstd{(m)} \hlkwb{<-} \hlstd{letters[}\hlnum{1}\hlopt{:}\hlstd{nr]}
\hlkwd{colnames}\hlstd{(m)} \hlkwb{<-} \hlstd{LETTERS[}\hlnum{1}\hlopt{:}\hlstd{nc]}
\hlstd{rg} \hlkwb{<-} \hlkwd{rep}\hlstd{(}\hlkwd{c}\hlstd{(}\hlstr{"foo"}\hlstd{,} \hlstr{"bar"}\hlstd{,} \hlstr{"baz"}\hlstd{),} \hlkwc{length.out}\hlstd{=nr)}
\hlstd{cg} \hlkwb{<-} \hlkwd{rep}\hlstd{(}\hlkwd{c}\hlstd{(}\hlstr{"Fuzzy"}\hlstd{,} \hlstr{"Busy"}\hlstd{),} \hlkwc{length.out}\hlstd{=nc)}
\hlstd{M} \hlkwb{<-} \hlkwd{cr_group}\hlstd{(m, rg, cg)}
\hlkwd{rm}\hlstd{(nr, nc)}
\end{alltt}
\end{kframe}
\end{knitrout}

\section{Examples}
Generate Table \ref{tab:1}:
\begin{kframe}
\begin{alltt}
\hlstd{dummy} \hlkwb{<-} \hlkwd{latex}\hlstd{(}\hlkwc{object}\hlstd{=M,} \hlkwc{caption}\hlstd{=}\hlstr{"Example 1: default settings"}\hlstd{,} \hlkwc{label}\hlstd{=}\hlstr{"tab:1"}\hlstd{)}
\end{alltt}
\end{kframe}%latex.default(object = new_object, rgroup = c("foo", "bar", "baz"),     n.rgroup = c(3, 2, 2), cgroup = c("Fuzzy", "Busy"), n.cgroup = c(3,         2), title = title, file = file, colheads = colheads,     ...)%
\begin{table}[!tbp]
\caption{Example 1: default settings\label{tab:1}} 
\begin{center}
\begin{tabular}{lrrrcrr}
\hline\hline
\multicolumn{1}{l}{\bfseries }&\multicolumn{3}{c}{\bfseries Fuzzy}&\multicolumn{1}{c}{\bfseries }&\multicolumn{2}{c}{\bfseries Busy}\tabularnewline
\cline{2-4} \cline{6-7}
\multicolumn{1}{l}{}&\multicolumn{1}{c}{A}&\multicolumn{1}{c}{B}&\multicolumn{1}{c}{C}&\multicolumn{1}{c}{}&\multicolumn{1}{c}{D}&\multicolumn{1}{c}{E}\tabularnewline
\hline
{\bfseries foo}&&&&&&\tabularnewline
~~a&$ 1$&$ 3$&$ 5$&&$ 2$&$ 4$\tabularnewline
~~d&$16$&$18$&$20$&&$17$&$19$\tabularnewline
~~g&$31$&$33$&$35$&&$32$&$34$\tabularnewline
\hline
{\bfseries bar}&&&&&&\tabularnewline
~~b&$ 6$&$ 8$&$10$&&$ 7$&$ 9$\tabularnewline
~~e&$21$&$23$&$25$&&$22$&$24$\tabularnewline
\hline
{\bfseries baz}&&&&&&\tabularnewline
~~c&$11$&$13$&$15$&&$12$&$14$\tabularnewline
~~f&$26$&$28$&$30$&&$27$&$29$\tabularnewline
\hline
\end{tabular}\end{center}

\end{table}

Generate Table  \ref{tab:2}:
\begin{kframe}
\begin{alltt}
\hlstd{M2} \hlkwb{<-} \hlstd{M;} \hlkwd{attr}\hlstd{(M2,} \hlstr{"rgroup"}\hlstd{)} \hlkwb{<-} \hlkwa{NULL}
\hlstd{dummy} \hlkwb{<-} \hlkwd{latex}\hlstd{(M2,} \hlkwc{caption}\hlstd{=}\hlstr{"Example 2: no 'rgroup' attribute"}\hlstd{,} \hlkwc{label}\hlstd{=}\hlstr{"tab:2"}\hlstd{)}
\end{alltt}
\end{kframe}%latex.default(object = new_object, cgroup = c("Fuzzy", "Busy"),     n.cgroup = c(3, 2), title = title, file = file, colheads = colheads,     ...)%
\begin{table}[!tbp]
\caption{Example 2: no 'rgroup' attribute\label{tab:2}} 
\begin{center}
\begin{tabular}{lrrrcrr}
\hline\hline
\multicolumn{1}{l}{\bfseries }&\multicolumn{3}{c}{\bfseries Fuzzy}&\multicolumn{1}{c}{\bfseries }&\multicolumn{2}{c}{\bfseries Busy}\tabularnewline
\cline{2-4} \cline{6-7}
\multicolumn{1}{l}{}&\multicolumn{1}{c}{A}&\multicolumn{1}{c}{B}&\multicolumn{1}{c}{C}&\multicolumn{1}{c}{}&\multicolumn{1}{c}{D}&\multicolumn{1}{c}{E}\tabularnewline
\hline
a&$ 1$&$ 3$&$ 5$&&$ 2$&$ 4$\tabularnewline
b&$ 6$&$ 8$&$10$&&$ 7$&$ 9$\tabularnewline
c&$11$&$13$&$15$&&$12$&$14$\tabularnewline
d&$16$&$18$&$20$&&$17$&$19$\tabularnewline
e&$21$&$23$&$25$&&$22$&$24$\tabularnewline
f&$26$&$28$&$30$&&$27$&$29$\tabularnewline
g&$31$&$33$&$35$&&$32$&$34$\tabularnewline
\hline
\end{tabular}\end{center}

\end{table}

Generate Table  \ref{tab:3}:
\begin{kframe}
\begin{alltt}
\hlstd{M2} \hlkwb{<-} \hlstd{M;} \hlkwd{attr}\hlstd{(M2,} \hlstr{"cgroup"}\hlstd{)} \hlkwb{<-} \hlkwa{NULL}
\hlstd{dummy} \hlkwb{<-} \hlkwd{latex}\hlstd{(M2,} \hlkwc{caption}\hlstd{=}\hlstr{"Example 3: no 'cgroup' attribute"}\hlstd{,} \hlkwc{label}\hlstd{=}\hlstr{"tab:3"}\hlstd{)}
\end{alltt}
\end{kframe}%latex.default(object = new_object, rgroup = c("foo", "bar", "baz"),     n.rgroup = c(3, 2, 2), title = title, file = file, colheads = colheads,     ...)%
\begin{table}[!tbp]
\caption{Example 3: no 'cgroup' attribute\label{tab:3}} 
\begin{center}
\begin{tabular}{lrrrrr}
\hline\hline
\multicolumn{1}{l}{}&\multicolumn{1}{c}{A}&\multicolumn{1}{c}{B}&\multicolumn{1}{c}{C}&\multicolumn{1}{c}{D}&\multicolumn{1}{c}{E}\tabularnewline
\hline
{\bfseries foo}&&&&&\tabularnewline
~~a&$ 1$&$ 2$&$ 3$&$ 4$&$ 5$\tabularnewline
~~d&$16$&$17$&$18$&$19$&$20$\tabularnewline
~~g&$31$&$32$&$33$&$34$&$35$\tabularnewline
\hline
{\bfseries bar}&&&&&\tabularnewline
~~b&$ 6$&$ 7$&$ 8$&$ 9$&$10$\tabularnewline
~~e&$21$&$22$&$23$&$24$&$25$\tabularnewline
\hline
{\bfseries baz}&&&&&\tabularnewline
~~c&$11$&$12$&$13$&$14$&$15$\tabularnewline
~~f&$26$&$27$&$28$&$29$&$30$\tabularnewline
\hline
\end{tabular}\end{center}

\end{table}

Generate Table  \ref{tab:4}:
\begin{kframe}
\begin{alltt}
\hlstd{dummy} \hlkwb{<-} \hlkwd{latex}\hlstd{(M,} \hlkwc{r.perm}\hlstd{=}\hlstr{'alphabetical'}\hlstd{,} \hlkwc{caption}\hlstd{=}\hlstr{"Example 4: rgroup alphabetically"}\hlstd{,} \hlkwc{label}\hlstd{=}\hlstr{"tab:4"}\hlstd{)}
\end{alltt}
\end{kframe}%latex.default(object = new_object, rgroup = c("bar", "baz", "foo"),     n.rgroup = c(2, 2, 3), cgroup = c("Fuzzy", "Busy"), n.cgroup = c(3,         2), title = title, file = file, colheads = colheads,     ...)%
\begin{table}[!tbp]
\caption{Example 4: rgroup alphabetically\label{tab:4}} 
\begin{center}
\begin{tabular}{lrrrcrr}
\hline\hline
\multicolumn{1}{l}{\bfseries }&\multicolumn{3}{c}{\bfseries Fuzzy}&\multicolumn{1}{c}{\bfseries }&\multicolumn{2}{c}{\bfseries Busy}\tabularnewline
\cline{2-4} \cline{6-7}
\multicolumn{1}{l}{}&\multicolumn{1}{c}{A}&\multicolumn{1}{c}{B}&\multicolumn{1}{c}{C}&\multicolumn{1}{c}{}&\multicolumn{1}{c}{D}&\multicolumn{1}{c}{E}\tabularnewline
\hline
{\bfseries bar}&&&&&&\tabularnewline
~~b&$ 6$&$ 8$&$10$&&$ 7$&$ 9$\tabularnewline
~~e&$21$&$23$&$25$&&$22$&$24$\tabularnewline
\hline
{\bfseries baz}&&&&&&\tabularnewline
~~c&$11$&$13$&$15$&&$12$&$14$\tabularnewline
~~f&$26$&$28$&$30$&&$27$&$29$\tabularnewline
\hline
{\bfseries foo}&&&&&&\tabularnewline
~~a&$ 1$&$ 3$&$ 5$&&$ 2$&$ 4$\tabularnewline
~~d&$16$&$18$&$20$&&$17$&$19$\tabularnewline
~~g&$31$&$33$&$35$&&$32$&$34$\tabularnewline
\hline
\end{tabular}\end{center}

\end{table}

Generate Table  \ref{tab:4}:
\begin{kframe}
\begin{alltt}
\hlstd{dummy} \hlkwb{<-} \hlkwd{latex}\hlstd{(M,} \hlkwc{r.perm}\hlstd{=}\hlkwd{c}\hlstd{(}\hlnum{3}\hlstd{,}\hlnum{2}\hlstd{,}\hlnum{1}\hlstd{),} \hlkwc{c.perm}\hlstd{=}\hlstr{'alphabetical'} \hlstd{,} \hlkwc{caption}\hlstd{=}\hlstr{"Example 5: cgroup alphabetically, rgroup permutated"}\hlstd{,} \hlkwc{label}\hlstd{=}\hlstr{"tab:5"}\hlstd{)}
\end{alltt}
\end{kframe}%latex.default(object = new_object, rgroup = c("foo", "baz", "bar"),     n.rgroup = c(3, 2, 2), cgroup = c("Busy", "Fuzzy"), n.cgroup = c(2,         3), title = title, file = file, colheads = colheads,     ...)%
\begin{table}[!tbp]
\caption{Example 5: cgroup alphabetically, rgroup permutated\label{tab:5}} 
\begin{center}
\begin{tabular}{lrrcrrr}
\hline\hline
\multicolumn{1}{l}{\bfseries }&\multicolumn{2}{c}{\bfseries Busy}&\multicolumn{1}{c}{\bfseries }&\multicolumn{3}{c}{\bfseries Fuzzy}\tabularnewline
\cline{2-3} \cline{5-7}
\multicolumn{1}{l}{}&\multicolumn{1}{c}{A}&\multicolumn{1}{c}{B}&\multicolumn{1}{c}{}&\multicolumn{1}{c}{C}&\multicolumn{1}{c}{D}&\multicolumn{1}{c}{E}\tabularnewline
\hline
{\bfseries foo}&&&&&&\tabularnewline
~~a&$ 2$&$ 4$&&$ 1$&$ 3$&$ 5$\tabularnewline
~~d&$17$&$19$&&$16$&$18$&$20$\tabularnewline
~~g&$32$&$34$&&$31$&$33$&$35$\tabularnewline
\hline
{\bfseries baz}&&&&&&\tabularnewline
~~c&$12$&$14$&&$11$&$13$&$15$\tabularnewline
~~f&$27$&$29$&&$26$&$28$&$30$\tabularnewline
\hline
{\bfseries bar}&&&&&&\tabularnewline
~~b&$ 7$&$ 9$&&$ 6$&$ 8$&$10$\tabularnewline
~~e&$22$&$24$&&$21$&$23$&$25$\tabularnewline
\hline
\end{tabular}\end{center}

\end{table}



\section{Using \texttt{cr\_groups} to make \texttt{cr\_groups} objects}
\begin{knitrout}
\definecolor{shadecolor}{rgb}{0.969, 0.969, 0.969}\color{fgcolor}\begin{kframe}
\begin{alltt}
\hlstd{n} \hlkwb{<-} \hlnum{7}
\hlstd{DF} \hlkwb{<-} \hlkwd{data.frame}\hlstd{(}
   \hlkwc{x.man}\hlstd{=}\hlnum{1}\hlopt{:}\hlstd{n,}
   \hlkwc{x.fem}\hlstd{=n}\hlopt{:}\hlnum{1}\hlstd{,}
   \hlkwc{y.man}\hlstd{=}\hlopt{-}\hlstd{(}\hlnum{1}\hlopt{:}\hlstd{n),}
   \hlkwc{y.fem}\hlstd{=}\hlopt{-}\hlstd{(n}\hlopt{:}\hlnum{1}\hlstd{)}
\hlstd{)}
\hlkwd{rownames}\hlstd{(DF)} \hlkwb{<-} \hlkwd{sprintf}\hlstd{(}\hlstr{"Ind: %d"}\hlstd{,} \hlnum{1}\hlopt{:}\hlstd{n)}
\hlstd{crDF} \hlkwb{<-} \hlkwd{cr_group}\hlstd{(}
   \hlkwc{x}\hlstd{=DF,}
   \hlkwc{rgroup}\hlstd{=}\hlkwd{rep}\hlstd{(LETTERS[}\hlnum{1}\hlopt{:}\hlnum{3}\hlstd{],}\hlkwc{len}\hlstd{=}\hlnum{7}\hlstd{),}
   \hlkwc{cgroup}\hlstd{=}\hlkwd{rep}\hlstd{(}\hlkwd{c}\hlstd{(}\hlstr{"gr-A"}\hlstd{,} \hlstr{"gr-B"}\hlstd{),} \hlkwc{each}\hlstd{=}\hlnum{2}\hlstd{),}
   \hlkwc{colnames} \hlstd{=} \hlkwd{gsub}\hlstd{(}\hlstr{"[x|y]\textbackslash{}\textbackslash{}."}\hlstd{,} \hlstr{""}\hlstd{,} \hlkwd{names}\hlstd{(DF))}
   \hlstd{)}

\hlkwd{str}\hlstd{(crDF)}
\end{alltt}
\begin{verbatim}
## An object of class 'cr_group' which contains:
## 'data.frame':	7 obs. of  4 variables:
##  $ x.man: int  1 2 3 4 5 6 7
##  $ x.fem: int  7 6 5 4 3 2 1
##  $ y.man: int  -1 -2 -3 -4 -5 -6 -7
##  $ y.fem: int  -7 -6 -5 -4 -3 -2 -1
##  - attr(*, "rgroup")= chr  "A" "B" "C" "A" ...
##  - attr(*, "cgroup")= chr  "gr-A" "gr-A" "gr-B" "gr-B"
##  - attr(*, "colnames")= chr  "man" "fem" "man" "fem"
\end{verbatim}
\begin{alltt}
\hlkwd{str}\hlstd{(crDF[}\hlnum{7}\hlopt{:}\hlnum{1}\hlstd{,}\hlkwd{c}\hlstd{(}\hlnum{1}\hlstd{,}\hlnum{3}\hlstd{,}\hlnum{2}\hlstd{,}\hlnum{4}\hlstd{)])}
\end{alltt}
\begin{verbatim}
## An object of class 'cr_group' which contains:
## 'data.frame':	7 obs. of  4 variables:
##  $ x.man: int  7 6 5 4 3 2 1
##  $ y.man: int  -7 -6 -5 -4 -3 -2 -1
##  $ x.fem: int  1 2 3 4 5 6 7
##  $ y.fem: int  -1 -2 -3 -4 -5 -6 -7
##  - attr(*, "rgroup")= chr  "A" "C" "B" "A" ...
##  - attr(*, "cgroup")= chr  "gr-A" "gr-B" "gr-A" "gr-B"
##  - attr(*, "colnames")= chr  "man" "man" "fem" "fem"
\end{verbatim}
\begin{alltt}
\hlstd{M} \hlkwb{<-} \hlkwd{matrix}\hlstd{(}\hlnum{1}\hlopt{:}\hlnum{12}\hlstd{,} \hlkwc{nrow}\hlstd{=}\hlnum{4}\hlstd{)}
\hlkwd{rownames}\hlstd{(M)} \hlkwb{<-} \hlkwd{sprintf}\hlstd{(}\hlstr{"Row %d"}\hlstd{,} \hlnum{1}\hlopt{:}\hlnum{4}\hlstd{)}
\hlkwd{colnames}\hlstd{(M)} \hlkwb{<-} \hlkwd{sprintf}\hlstd{(}\hlstr{"Col %d"}\hlstd{,} \hlnum{1}\hlopt{:}\hlnum{3}\hlstd{)}
\hlstd{(crM} \hlkwb{<-} \hlkwd{cr_group}\hlstd{(}\hlkwc{x}\hlstd{=M,} \hlkwc{rgroup}\hlstd{=LETTERS[}\hlnum{1}\hlopt{:}\hlnum{4}\hlstd{],} \hlkwc{cgroup}\hlstd{=letters[}\hlnum{1}\hlopt{:}\hlnum{3}\hlstd{]))}
\end{alltt}
\begin{verbatim}
##       Col 1 Col 2 Col 3
## Row 1     1     5     9
## Row 2     2     6    10
## Row 3     3     7    11
## Row 4     4     8    12
## attr(,"rgroup")
## [1] "A" "B" "C" "D"
## attr(,"cgroup")
## [1] "a" "b" "c"
## attr(,"class")
## [1] "cr_group" "matrix"
\end{verbatim}
\begin{alltt}
\hlstd{crM[}\hlnum{4}\hlopt{:}\hlnum{1}\hlstd{,}\hlkwd{c}\hlstd{(}\hlnum{3}\hlstd{,}\hlnum{1}\hlstd{,}\hlnum{2}\hlstd{)]}
\end{alltt}
\begin{verbatim}
##       Col 3 Col 1 Col 2
## Row 4    12     4     8
## Row 3    11     3     7
## Row 2    10     2     6
## Row 1     9     1     5
## attr(,"rgroup")
## [1] "D" "C" "B" "A"
## attr(,"cgroup")
## [1] "c" "a" "b"
## attr(,"class")
## [1] "cr_group" "matrix"
\end{verbatim}
\end{kframe}
\end{knitrout}

\begin{kframe}
\begin{alltt}
\hlstd{dummy} \hlkwb{<-} \hlkwd{latex}\hlstd{(crDF,} \hlkwc{caption}\hlstd{=}\hlstr{"'latex' on 'crDF'"}\hlstd{,} \hlkwc{label}\hlstd{=}\hlstr{"tab:a1"}\hlstd{,} \hlkwc{colheads}\hlstd{=}\hlnum{TRUE}\hlstd{)}
\end{alltt}
\end{kframe}%latex.default(object = new_object, rgroup = c("A", "B", "C"),     n.rgroup = c(3, 2, 2), cgroup = c("gr-A", "gr-B"), n.cgroup = c(2,         2), title = title, file = file, colheads = colheads,     ...)%
\begin{table}[!tbp]
\caption{'latex' on 'crDF'\label{tab:a1}} 
\begin{center}
\begin{tabular}{lrrcrr}
\hline\hline
\multicolumn{1}{l}{\bfseries }&\multicolumn{2}{c}{\bfseries gr-A}&\multicolumn{1}{c}{\bfseries }&\multicolumn{2}{c}{\bfseries gr-B}\tabularnewline
\cline{2-3} \cline{5-6}
\multicolumn{1}{l}{}&\multicolumn{1}{c}{man}&\multicolumn{1}{c}{fem}&\multicolumn{1}{c}{}&\multicolumn{1}{c}{man}&\multicolumn{1}{c}{fem}\tabularnewline
\hline
{\bfseries A}&&&&&\tabularnewline
~~Ind: 1&$1$&$7$&&$-1$&$-7$\tabularnewline
~~Ind: 4&$4$&$4$&&$-4$&$-4$\tabularnewline
~~Ind: 7&$7$&$1$&&$-7$&$-1$\tabularnewline
\hline
{\bfseries B}&&&&&\tabularnewline
~~Ind: 2&$2$&$6$&&$-2$&$-6$\tabularnewline
~~Ind: 5&$5$&$3$&&$-5$&$-3$\tabularnewline
\hline
{\bfseries C}&&&&&\tabularnewline
~~Ind: 3&$3$&$5$&&$-3$&$-5$\tabularnewline
~~Ind: 6&$6$&$2$&&$-6$&$-2$\tabularnewline
\hline
\end{tabular}\end{center}

\end{table}
\begin{kframe}\begin{alltt}
\hlstd{dummy} \hlkwb{<-} \hlkwd{latex}\hlstd{(crM,} \hlkwc{caption}\hlstd{=}\hlstr{"'latex' on 'crM'"}\hlstd{,} \hlkwc{label}\hlstd{=}\hlstr{"tab:a2"}\hlstd{,} \hlkwc{colheads}\hlstd{=}\hlnum{TRUE}\hlstd{)}
\end{alltt}
\end{kframe}%latex.default(object = new_object, rgroup = c("A", "B", "C",     "D"), n.rgroup = c(1, 1, 1, 1), cgroup = c("a", "b", "c"),     n.cgroup = c(1, 1, 1), title = title, file = file, colheads = colheads,     ...)%
\begin{table}[!tbp]
\caption{'latex' on 'crM'\label{tab:a2}} 
\begin{center}
\begin{tabular}{lrcrcr}
\hline\hline
\multicolumn{1}{l}{\bfseries }&\multicolumn{1}{c}{\bfseries a}&\multicolumn{1}{c}{\bfseries }&\multicolumn{1}{c}{\bfseries b}&\multicolumn{1}{c}{\bfseries }&\multicolumn{1}{c}{\bfseries c}\tabularnewline
\cline{2-2} \cline{4-4} \cline{6-6}
\multicolumn{1}{l}{}&\multicolumn{1}{c}{Col 1}&\multicolumn{1}{c}{}&\multicolumn{1}{c}{Col 2}&\multicolumn{1}{c}{}&\multicolumn{1}{c}{Col 3}\tabularnewline
\hline
{\bfseries A}&&&&&\tabularnewline
~~Row 1&$1$&&$5$&&$ 9$\tabularnewline
\hline
{\bfseries B}&&&&&\tabularnewline
~~Row 2&$2$&&$6$&&$10$\tabularnewline
\hline
{\bfseries C}&&&&&\tabularnewline
~~Row 3&$3$&&$7$&&$11$\tabularnewline
\hline
{\bfseries D}&&&&&\tabularnewline
~~Row 4&$4$&&$8$&&$12$\tabularnewline
\hline
\end{tabular}\end{center}

\end{table}


\section{Errors, warnings, and such}
\textbf{N.B.} To pass the R CMD check, the code of a vignette must be able to run.
Therefore, to illustrate errors, one must 'catch' them (see below).

\noindent Generate an error:
\begin{knitrout}
\definecolor{shadecolor}{rgb}{0.969, 0.969, 0.969}\color{fgcolor}\begin{kframe}
\begin{alltt}
\hlstd{M2} \hlkwb{<-} \hlstd{crM;} \hlkwd{attr}\hlstd{(M2,} \hlstr{"cgroup"}\hlstd{)} \hlkwb{<-} \hlkwa{NULL}\hlstd{;} \hlkwd{attr}\hlstd{(M2,} \hlstr{"rgroup"}\hlstd{)} \hlkwb{<-} \hlkwa{NULL}
\hlkwd{tryCatch}\hlstd{(dummy} \hlkwb{<-} \hlkwd{latex}\hlstd{(M2),} \hlkwc{error} \hlstd{=} \hlkwa{function}\hlstd{(}\hlkwc{e}\hlstd{)} \hlkwd{print}\hlstd{(e))}
\end{alltt}
\end{kframe}
\end{knitrout}
\noindent Generate an error:
\begin{knitrout}
\definecolor{shadecolor}{rgb}{0.969, 0.969, 0.969}\color{fgcolor}\begin{kframe}
\begin{alltt}
\hlstd{M2} \hlkwb{<-} \hlstd{crM;} \hlkwd{class}\hlstd{(M2)} \hlkwb{<-} \hlkwa{NULL}
\hlkwd{tryCatch}\hlstd{(dummy} \hlkwb{<-} \hlkwd{latex.cr_group}\hlstd{(M2),} \hlkwc{error} \hlstd{=} \hlkwa{function}\hlstd{(}\hlkwc{e}\hlstd{)} \hlkwd{print}\hlstd{(e))}
\end{alltt}
\end{kframe}
\end{knitrout}
\noindent Generate a message (output hidden):
\begin{knitrout}
\definecolor{shadecolor}{rgb}{0.969, 0.969, 0.969}\color{fgcolor}\begin{kframe}
\begin{alltt}
\hlstd{dummy} \hlkwb{<-} \hlkwd{latex}\hlstd{(crM,} \hlkwc{r.perm}\hlstd{=}\hlstr{'as'}\hlstd{)}
\end{alltt}


{\ttfamily\noindent\itshape\color{messagecolor}{\#\# [latex.cr\_group] 'rperm' interpreted as 'as.is'}}\end{kframe}
\end{knitrout}

\end{document}
