%\VignetteEngine{knitr::knitr}
%\VignetteIndexEntry{How to use 'refactor' (knitr produced documentation)}
\documentclass{article}\usepackage[]{graphicx}\usepackage[]{color}
%% maxwidth is the original width if it is less than linewidth
%% otherwise use linewidth (to make sure the graphics do not exceed the margin)
\makeatletter
\def\maxwidth{ %
  \ifdim\Gin@nat@width>\linewidth
    \linewidth
  \else
    \Gin@nat@width
  \fi
}
\makeatother

\definecolor{fgcolor}{rgb}{0.345, 0.345, 0.345}
\newcommand{\hlnum}[1]{\textcolor[rgb]{0.686,0.059,0.569}{#1}}%
\newcommand{\hlstr}[1]{\textcolor[rgb]{0.192,0.494,0.8}{#1}}%
\newcommand{\hlcom}[1]{\textcolor[rgb]{0.678,0.584,0.686}{\textit{#1}}}%
\newcommand{\hlopt}[1]{\textcolor[rgb]{0,0,0}{#1}}%
\newcommand{\hlstd}[1]{\textcolor[rgb]{0.345,0.345,0.345}{#1}}%
\newcommand{\hlkwa}[1]{\textcolor[rgb]{0.161,0.373,0.58}{\textbf{#1}}}%
\newcommand{\hlkwb}[1]{\textcolor[rgb]{0.69,0.353,0.396}{#1}}%
\newcommand{\hlkwc}[1]{\textcolor[rgb]{0.333,0.667,0.333}{#1}}%
\newcommand{\hlkwd}[1]{\textcolor[rgb]{0.737,0.353,0.396}{\textbf{#1}}}%

\usepackage{framed}
\makeatletter
\newenvironment{kframe}{%
 \def\at@end@of@kframe{}%
 \ifinner\ifhmode%
  \def\at@end@of@kframe{\end{minipage}}%
  \begin{minipage}{\columnwidth}%
 \fi\fi%
 \def\FrameCommand##1{\hskip\@totalleftmargin \hskip-\fboxsep
 \colorbox{shadecolor}{##1}\hskip-\fboxsep
     % There is no \\@totalrightmargin, so:
     \hskip-\linewidth \hskip-\@totalleftmargin \hskip\columnwidth}%
 \MakeFramed {\advance\hsize-\width
   \@totalleftmargin\z@ \linewidth\hsize
   \@setminipage}}%
 {\par\unskip\endMakeFramed%
 \at@end@of@kframe}
\makeatother

\definecolor{shadecolor}{rgb}{.97, .97, .97}
\definecolor{messagecolor}{rgb}{0, 0, 0}
\definecolor{warningcolor}{rgb}{1, 0, 1}
\definecolor{errorcolor}{rgb}{1, 0, 0}
\newenvironment{knitrout}{}{} % an empty environment to be redefined in TeX

\usepackage{alltt}




\title{refactor}
\author{Henrik Renlund}
\newcommand{\code}{\texttt}
\IfFileExists{upquote.sty}{\usepackage{upquote}}{}
\begin{document}
\maketitle
\tableofcontents

\section{Purpose}
The function\footnote{The name \code{relevel} was taken!} \code{refactor} was build mainly to be able to easily rename, reorder, and merge the levels of a factor. It can also create and exclude levels, and set \code{NA} to be a level.

Motivating example:
\begin{knitrout}
\definecolor{shadecolor}{rgb}{0.969, 0.969, 0.969}\color{fgcolor}\begin{kframe}
\begin{alltt}
\hlstd{lev} \hlkwb{<-} \hlkwd{c}\hlstd{(}\hlstr{" "}\hlstd{,} \hlstr{"never"}\hlstd{,} \hlstr{"some"}\hlstd{,} \hlstr{"a lot"}\hlstd{)}
\hlstd{smoking} \hlkwb{<-} \hlkwd{factor}\hlstd{(}\hlkwd{sample}\hlstd{(}\hlkwc{x}\hlstd{=lev,} \hlnum{100}\hlstd{,} \hlnum{TRUE}\hlstd{),} \hlkwc{levels}\hlstd{=lev)}
\hlkwd{table}\hlstd{(smoking)}
\end{alltt}
\begin{verbatim}
## smoking
##       never  some a lot 
##    28    23    27    22
\end{verbatim}
\end{kframe}
\end{knitrout}
Suppose you would want to clean up the variable \code{smoking} to have only levels "no" (previously 'never'), "yes" (previously 'some' and 'a lot') and "unknown" (previously ' ')  - and in that order.
\begin{knitrout}
\definecolor{shadecolor}{rgb}{0.969, 0.969, 0.969}\color{fgcolor}\begin{kframe}
\begin{alltt}
\hlstd{L} \hlkwb{<-} \hlkwd{list}\hlstd{(}\hlkwc{never}\hlstd{=}\hlstr{"no"}\hlstd{,} \hlkwc{some}\hlstd{=}\hlkwd{c}\hlstd{(}\hlstr{"yes"}\hlstd{,} \hlstr{"a lot"}\hlstd{),} \hlstr{" "}\hlstd{=}\hlstr{"unknown"}\hlstd{)}
\hlkwd{table}\hlstd{(}\hlkwd{refactor}\hlstd{(smoking, L))}
\end{alltt}
\begin{verbatim}
## 
##      no     yes unknown 
##      23      49      28
\end{verbatim}
\end{kframe}
\end{knitrout}

\section{Basic functionality}

\subsection{Renaming and reordering levels}
Levels can be renamed by providing a list with entries of the old level name that equals a new name.
\begin{knitrout}
\definecolor{shadecolor}{rgb}{0.969, 0.969, 0.969}\color{fgcolor}\begin{kframe}
\begin{alltt}
\hlstd{(x} \hlkwb{<-} \hlkwd{factor}\hlstd{(LETTERS[}\hlnum{1}\hlopt{:}\hlnum{2}\hlstd{]))}
\end{alltt}
\begin{verbatim}
## [1] A B
## Levels: A B
\end{verbatim}
\begin{alltt}
\hlstd{L} \hlkwb{<-} \hlkwd{list}\hlstd{(}\hlkwc{A}\hlstd{=}\hlstr{"newA"}\hlstd{,} \hlkwc{B}\hlstd{=}\hlstr{"newB"}\hlstd{)}
\hlkwd{refactor}\hlstd{(x,L)}
\end{alltt}
\begin{verbatim}
## [1] newA newB
## Levels: newA newB
\end{verbatim}
\end{kframe}
\end{knitrout}
The order of the list determines the order of the levels created.
\begin{knitrout}
\definecolor{shadecolor}{rgb}{0.969, 0.969, 0.969}\color{fgcolor}\begin{kframe}
\begin{alltt}
\hlstd{(x} \hlkwb{<-} \hlkwd{factor}\hlstd{(LETTERS[}\hlnum{1}\hlopt{:}\hlnum{2}\hlstd{],} \hlkwc{levels}\hlstd{=LETTERS[}\hlnum{2}\hlopt{:}\hlnum{1}\hlstd{]))}
\end{alltt}
\begin{verbatim}
## [1] A B
## Levels: B A
\end{verbatim}
\begin{alltt}
\hlstd{L} \hlkwb{<-} \hlkwd{list}\hlstd{(}\hlkwc{A}\hlstd{=}\hlstr{"newA"}\hlstd{,} \hlkwc{B}\hlstd{=}\hlstr{"newB"}\hlstd{)}
\hlkwd{refactor}\hlstd{(x,L)}
\end{alltt}
\begin{verbatim}
## [1] newA newB
## Levels: newA newB
\end{verbatim}
\end{kframe}
\end{knitrout}
If you simply want to reorder you must provide a \code{NULL} new name.
\begin{knitrout}
\definecolor{shadecolor}{rgb}{0.969, 0.969, 0.969}\color{fgcolor}\begin{kframe}
\begin{alltt}
\hlstd{(x} \hlkwb{<-} \hlkwd{factor}\hlstd{(LETTERS[}\hlnum{1}\hlopt{:}\hlnum{3}\hlstd{]))}
\end{alltt}
\begin{verbatim}
## [1] A B C
## Levels: A B C
\end{verbatim}
\begin{alltt}
\hlstd{L} \hlkwb{<-} \hlkwd{list}\hlstd{(}\hlkwc{B}\hlstd{=}\hlkwa{NULL}\hlstd{,} \hlkwc{A}\hlstd{=}\hlkwa{NULL}\hlstd{,} \hlkwc{C}\hlstd{=}\hlkwa{NULL}\hlstd{)}
\hlkwd{refactor}\hlstd{(x,L)}
\end{alltt}
\begin{verbatim}
## [1] A B C
## Levels: B A C
\end{verbatim}
\end{kframe}
\end{knitrout}
The list \code{L} need not cover all existing levels. If not, the remaining levels will be added in the same order as they appeared in the original.
\begin{knitrout}
\definecolor{shadecolor}{rgb}{0.969, 0.969, 0.969}\color{fgcolor}\begin{kframe}
\begin{alltt}
\hlstd{(x} \hlkwb{<-} \hlkwd{factor}\hlstd{(LETTERS[}\hlnum{1}\hlopt{:}\hlnum{5}\hlstd{]))}
\end{alltt}
\begin{verbatim}
## [1] A B C D E
## Levels: A B C D E
\end{verbatim}
\begin{alltt}
\hlstd{L} \hlkwb{<-} \hlkwd{list}\hlstd{(}\hlkwc{D}\hlstd{=}\hlkwa{NULL}\hlstd{,} \hlkwc{B}\hlstd{=}\hlstr{"newB"}\hlstd{)}
\hlkwd{refactor}\hlstd{(x,L)}
\end{alltt}
\begin{verbatim}
## [1] A    newB C    D    E   
## Levels: D newB A C E
\end{verbatim}
\end{kframe}
\end{knitrout}
\subsection{Adding and merging levels}
If \code{L} includes a non-existing level, this level will be added.
\begin{knitrout}
\definecolor{shadecolor}{rgb}{0.969, 0.969, 0.969}\color{fgcolor}\begin{kframe}
\begin{alltt}
\hlstd{(x} \hlkwb{<-} \hlkwd{factor}\hlstd{(LETTERS[}\hlnum{1}\hlopt{:}\hlnum{2}\hlstd{]))}
\end{alltt}
\begin{verbatim}
## [1] A B
## Levels: A B
\end{verbatim}
\begin{alltt}
\hlstd{L} \hlkwb{<-} \hlkwd{list}\hlstd{(}\hlkwc{Foo}\hlstd{=}\hlkwa{NULL}\hlstd{)}
\hlkwd{refactor}\hlstd{(x,L)}
\end{alltt}
\begin{verbatim}
## [1] A B
## Levels: Foo A B
\end{verbatim}
\end{kframe}
\end{knitrout}
There is an option to put these new levels at the end of the levels created.
\begin{knitrout}
\definecolor{shadecolor}{rgb}{0.969, 0.969, 0.969}\color{fgcolor}\begin{kframe}
\begin{alltt}
\hlstd{(x} \hlkwb{<-} \hlkwd{factor}\hlstd{(LETTERS[}\hlnum{1}\hlopt{:}\hlnum{2}\hlstd{]))}
\end{alltt}
\begin{verbatim}
## [1] A B
## Levels: A B
\end{verbatim}
\begin{alltt}
\hlstd{L} \hlkwb{<-} \hlkwd{list}\hlstd{(}\hlkwc{Foo}\hlstd{=}\hlkwa{NULL}\hlstd{,} \hlkwc{Bar}\hlstd{=}\hlkwa{NULL}\hlstd{)}
\hlkwd{refactor}\hlstd{(x,L,} \hlkwc{new.last}\hlstd{=}\hlnum{TRUE}\hlstd{)}
\end{alltt}
\begin{verbatim}
## [1] A B
## Levels: A B Foo Bar
\end{verbatim}
\end{kframe}
\end{knitrout}
If you want to merge two levels this can be acheived by including one in the other, as is best shown by example:
\begin{knitrout}
\definecolor{shadecolor}{rgb}{0.969, 0.969, 0.969}\color{fgcolor}\begin{kframe}
\begin{alltt}
\hlstd{(x} \hlkwb{<-} \hlkwd{factor}\hlstd{(LETTERS[}\hlnum{1}\hlopt{:}\hlnum{3}\hlstd{]))}
\end{alltt}
\begin{verbatim}
## [1] A B C
## Levels: A B C
\end{verbatim}
\begin{alltt}
\hlstd{L} \hlkwb{<-} \hlkwd{list}\hlstd{(}\hlkwc{A}\hlstd{=}\hlstr{"B"}\hlstd{)} \hlcom{# level B to be absorbed by level A}
\hlkwd{refactor}\hlstd{(x,L)}
\end{alltt}
\begin{verbatim}
## [1] A A C
## Levels: A C
\end{verbatim}
\end{kframe}
\end{knitrout}
Note that the above did not rename \code{A} as \code{B}. The entry \code{A} in list \code{L} can contain a new name as well as levels to be absorbed. Thus it cannot be renamed as an existing level.
\begin{knitrout}
\definecolor{shadecolor}{rgb}{0.969, 0.969, 0.969}\color{fgcolor}\begin{kframe}
\begin{alltt}
\hlstd{(x} \hlkwb{<-} \hlkwd{factor}\hlstd{(LETTERS[}\hlnum{1}\hlopt{:}\hlnum{3}\hlstd{]))}
\end{alltt}
\begin{verbatim}
## [1] A B C
## Levels: A B C
\end{verbatim}
\begin{alltt}
\hlstd{L} \hlkwb{<-} \hlkwd{list}\hlstd{(}\hlkwc{A}\hlstd{=}\hlkwd{c}\hlstd{(}\hlstr{"merge_AB"}\hlstd{,} \hlstr{"B"}\hlstd{))}
\hlkwd{refactor}\hlstd{(x,L)}
\end{alltt}
\begin{verbatim}
## [1] merge_AB merge_AB C       
## Levels: merge_AB C
\end{verbatim}
\end{kframe}
\end{knitrout}
The same thing can be acheived by
\begin{knitrout}
\definecolor{shadecolor}{rgb}{0.969, 0.969, 0.969}\color{fgcolor}\begin{kframe}
\begin{alltt}
\hlstd{(x} \hlkwb{<-} \hlkwd{factor}\hlstd{(LETTERS[}\hlnum{1}\hlopt{:}\hlnum{3}\hlstd{]))}
\end{alltt}
\begin{verbatim}
## [1] A B C
## Levels: A B C
\end{verbatim}
\begin{alltt}
\hlstd{L} \hlkwb{<-} \hlkwd{list}\hlstd{(}\hlkwc{B}\hlstd{=}\hlkwd{c}\hlstd{(}\hlstr{"merge_AB"}\hlstd{,} \hlstr{"A"}\hlstd{))}
\hlkwd{refactor}\hlstd{(x,L)}
\end{alltt}
\begin{verbatim}
## [1] merge_AB merge_AB C       
## Levels: merge_AB C
\end{verbatim}
\end{kframe}
\end{knitrout}
The order of entry \code{B} is unimportant, but the first non-level name will be used as a new name.
\begin{knitrout}
\definecolor{shadecolor}{rgb}{0.969, 0.969, 0.969}\color{fgcolor}\begin{kframe}
\begin{alltt}
\hlstd{(x} \hlkwb{<-} \hlkwd{factor}\hlstd{(LETTERS[}\hlnum{1}\hlopt{:}\hlnum{3}\hlstd{]))}
\end{alltt}
\begin{verbatim}
## [1] A B C
## Levels: A B C
\end{verbatim}
\begin{alltt}
\hlstd{L} \hlkwb{<-} \hlkwd{list}\hlstd{(}\hlkwc{B}\hlstd{=}\hlkwd{c}\hlstd{(}\hlstr{"A"}\hlstd{,} \hlstr{"merger"}\hlstd{,} \hlstr{"new_name"}\hlstd{))}
\hlkwd{refactor}\hlstd{(x,L)}
\end{alltt}


{\ttfamily\noindent\color{warningcolor}{\#\# Warning in refactor(x, L): [refactor] Multiple new names for level 'B'. Only the first one will be used.}}\begin{verbatim}
## [1] merger merger C     
## Levels: merger C
\end{verbatim}
\end{kframe}
\end{knitrout}
One cannot merge an existing level into more than one place, the first encountered will be used.
\begin{knitrout}
\definecolor{shadecolor}{rgb}{0.969, 0.969, 0.969}\color{fgcolor}\begin{kframe}
\begin{alltt}
\hlstd{(x} \hlkwb{<-} \hlkwd{factor}\hlstd{(LETTERS[}\hlnum{1}\hlopt{:}\hlnum{3}\hlstd{]))}
\end{alltt}
\begin{verbatim}
## [1] A B C
## Levels: A B C
\end{verbatim}
\begin{alltt}
\hlstd{L} \hlkwb{<-} \hlkwd{list}\hlstd{(}\hlkwc{B}\hlstd{=}\hlkwd{c}\hlstd{(}\hlstr{"merge_AB"}\hlstd{,}\hlstr{"A"}\hlstd{),} \hlkwc{C}\hlstd{=}\hlkwd{c}\hlstd{(}\hlstr{"new_C"}\hlstd{,} \hlstr{"A"}\hlstd{))}
\hlkwd{refactor}\hlstd{(x,L)}
\end{alltt}


{\ttfamily\noindent\color{warningcolor}{\#\# Warning in refactor(x, L): [refactor] Key 'C' has entries used previously. These will be skipped}}\begin{verbatim}
## [1] merge_AB merge_AB new_C   
## Levels: merge_AB new_C
\end{verbatim}
\end{kframe}
\end{knitrout}

\subsection{The empty string}
One cannot specify the element name in a list explicitly to be the empty string
\begin{knitrout}
\definecolor{shadecolor}{rgb}{0.969, 0.969, 0.969}\color{fgcolor}\begin{kframe}
\begin{alltt}
\hlcom{# list(a_name="value", ""="another_value") # not possible}
\hlkwd{list}\hlstd{(}\hlkwc{a_name}\hlstd{=}\hlstr{"value"}\hlstd{,} \hlstr{"another_value"}\hlstd{)} \hlcom{# possible}
\end{alltt}
\begin{verbatim}
## $a_name
## [1] "value"
## 
## [[2]]
## [1] "another_value"
\end{verbatim}
\end{kframe}
\end{knitrout}
Therefore, if you want to recode a empty string value can do
\begin{knitrout}
\definecolor{shadecolor}{rgb}{0.969, 0.969, 0.969}\color{fgcolor}\begin{kframe}
\begin{alltt}
\hlstd{x} \hlkwb{<-} \hlkwd{c}\hlstd{(}\hlstr{"A"}\hlstd{,} \hlstr{""}\hlstd{,} \hlstr{"B"}\hlstd{)}
\hlstd{L} \hlkwb{<-} \hlkwd{list}\hlstd{(}\hlstr{"A"}\hlstd{=}\hlstr{"New A"}\hlstd{,} \hlstr{"not_so_empty_now"}\hlstd{,} \hlstr{"B"}\hlstd{=}\hlstr{"New B"}\hlstd{)}
\hlkwd{refactor}\hlstd{(x, L)}
\end{alltt}
\begin{verbatim}
## [1] New A            not_so_empty_now New B           
## Levels: New A not_so_empty_now New B
\end{verbatim}
\end{kframe}
\end{knitrout}



\subsection{Excluding and adding \code{NA} as level}
You can delete levels, i.e.\ turning it into \code{NA}.
\begin{knitrout}
\definecolor{shadecolor}{rgb}{0.969, 0.969, 0.969}\color{fgcolor}\begin{kframe}
\begin{alltt}
\hlstd{(x} \hlkwb{<-} \hlkwd{factor}\hlstd{(LETTERS[}\hlnum{1}\hlopt{:}\hlnum{4}\hlstd{]))}
\end{alltt}
\begin{verbatim}
## [1] A B C D
## Levels: A B C D
\end{verbatim}
\begin{alltt}
\hlkwd{refactor}\hlstd{(x,}\hlkwc{exclude}\hlstd{=}\hlkwd{c}\hlstd{(}\hlstr{"A"}\hlstd{,}\hlstr{"C"}\hlstd{))}
\end{alltt}
\begin{verbatim}
## [1] <NA> B    <NA> D   
## Levels: B D
\end{verbatim}
\end{kframe}
\end{knitrout}
You can make \code{NA} into a level \code{missing} by setting \code{na.level} to any non null value
\begin{knitrout}
\definecolor{shadecolor}{rgb}{0.969, 0.969, 0.969}\color{fgcolor}\begin{kframe}
\begin{alltt}
\hlstd{(x} \hlkwb{<-} \hlkwd{factor}\hlstd{(}\hlkwd{c}\hlstd{(LETTERS[}\hlnum{1}\hlopt{:}\hlnum{2}\hlstd{],}\hlnum{NA}\hlstd{)))}
\end{alltt}
\begin{verbatim}
## [1] A    B    <NA>
## Levels: A B
\end{verbatim}
\begin{alltt}
\hlkwd{refactor}\hlstd{(x,}\hlkwc{na.level}\hlstd{=}\hlnum{TRUE}\hlstd{)}
\end{alltt}
\begin{verbatim}
## [1] A       B       missing
## Levels: A B missing
\end{verbatim}
\end{kframe}
\end{knitrout}
or specify any name by setting \code{na.level} to a string
\begin{knitrout}
\definecolor{shadecolor}{rgb}{0.969, 0.969, 0.969}\color{fgcolor}\begin{kframe}
\begin{alltt}
\hlstd{(x} \hlkwb{<-} \hlkwd{factor}\hlstd{(}\hlkwd{c}\hlstd{(LETTERS[}\hlnum{1}\hlopt{:}\hlnum{2}\hlstd{],}\hlnum{NA}\hlstd{)))}
\end{alltt}
\begin{verbatim}
## [1] A    B    <NA>
## Levels: A B
\end{verbatim}
\begin{alltt}
\hlkwd{refactor}\hlstd{(x,}\hlkwc{na.level}\hlstd{=}\hlstr{"saknas"}\hlstd{)}
\end{alltt}
\begin{verbatim}
## [1] A      B      saknas
## Levels: A B saknas
\end{verbatim}
\end{kframe}
\end{knitrout}

\section{Order of operations}
The function operates in the following order:
\begin{itemize}
\item if wanted, turn \code{NA} into a level
\item if wanted, turn certain levels into \code{NA}
\item if wanted, rename and reorder levels
\end{itemize}
Thus you can delete levels whilst making original \code{NA} into a new level.
\begin{knitrout}
\definecolor{shadecolor}{rgb}{0.969, 0.969, 0.969}\color{fgcolor}\begin{kframe}
\begin{alltt}
\hlstd{(x} \hlkwb{<-} \hlkwd{factor}\hlstd{(}\hlkwd{c}\hlstd{(LETTERS[}\hlnum{1}\hlopt{:}\hlnum{2}\hlstd{],}\hlnum{NA}\hlstd{)))}
\end{alltt}
\begin{verbatim}
## [1] A    B    <NA>
## Levels: A B
\end{verbatim}
\begin{alltt}
\hlkwd{refactor}\hlstd{(x,} \hlkwc{exclude}\hlstd{=}\hlstr{"A"}\hlstd{,} \hlkwc{na.level}\hlstd{=}\hlnum{TRUE}\hlstd{)}
\end{alltt}
\begin{verbatim}
## [1] <NA>    B       missing
## Levels: B missing
\end{verbatim}
\end{kframe}
\end{knitrout}
But would you want to consider both \code{A} and \code{NA} as \code{missing} you would have to 'rename' \code{NA} and let that name be absorbed by \code{A} which takes the new name 'missing'.
\begin{knitrout}
\definecolor{shadecolor}{rgb}{0.969, 0.969, 0.969}\color{fgcolor}\begin{kframe}
\begin{alltt}
\hlstd{(x} \hlkwb{<-} \hlkwd{factor}\hlstd{(}\hlkwd{c}\hlstd{(LETTERS[}\hlnum{1}\hlopt{:}\hlnum{2}\hlstd{],}\hlnum{NA}\hlstd{)))}
\end{alltt}
\begin{verbatim}
## [1] A    B    <NA>
## Levels: A B
\end{verbatim}
\begin{alltt}
\hlstd{L} \hlkwb{<-} \hlkwd{list}\hlstd{(}\hlkwc{A}\hlstd{=}\hlkwd{c}\hlstd{(}\hlstr{"missing"}\hlstd{,} \hlstr{"NA_name"}\hlstd{))}
\hlkwd{refactor}\hlstd{(x, L,} \hlkwc{na.level}\hlstd{=}\hlstr{"NA_name"}\hlstd{)}
\end{alltt}
\begin{verbatim}
## [1] missing B       missing
## Levels: missing B
\end{verbatim}
\end{kframe}
\end{knitrout}




\end{document}
